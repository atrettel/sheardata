% Copyright (C) 2020-2022 Andrew Trettel
%
% SPDX-License-Identifier: CC-BY-4.0
\chapter{Design of the database}


\section{Project goals}

The primary goal of this project is to create a single SQLite database that
contains data from as many different shear layer experiments and simulations as
possible.  The data should include uncertainties where possible, and the
project should also concentrate on ensuring that the data remain accessible and
understandable in the long term.  The secondary goal is to create a report
documenting the database and the data itself.  This report serves as a data and
literature review for many different flow classes (a meta-analysis of sorts).

Using this database, it should be possible to ask many different questions
about various kinds of flows directly, without requiring additional
experimentation or simulation.  It also should reveal gaps in current
knowledge, to inform new directions for research.  Lastly, the database would
be valuable for the validation of new theories and numerical codes.


\section{Database organization}

Unlike previous databases, this database does not store data as profiles or as
``header'' data.  Instead, it organizes data into four categories: study data,
series data, station data, and point data.  Each piece of data is stored as a
single record in one of those four categories, and this design allows for
greater flexibility and even redundancy at times, since the same variable can
be stored multiple times as different values, which could occur if different
measurement techniques or averaging systems were used.

Studies are the primary level of categorization.  Each study is a series of
observations by a research group, like a paper or a set of related papers.
Studies contain series, series contain stations, and stations contain points.
Series refer to a particular flow field in its entirety, like the entire flow
field around an object, or an entire set of boundary layer profiles.  Stations
refer to a single profile in space at specified streamwise and spanwise
coordinates.  And finally, points refer to particular data points in space with
all three coordinates specified.

Different quantities only exist within different categories of data.  For
example, consider the flow around an object.  The drag coefficient is a series
quantity, the boundary layer thickness is a station quantity, and the
streamwise velocity is a point quantity.  Some quantities only exist at walls,
like the skin friction coefficient, but since there could be multiple walls (as
in internal flows) these quantities are stored as point data, though in many
cases they can be identified uniquely by specifying the station.

To locate records in the database, each category of data is assigned a unique
identifier of the following form: flow class (1 letter), year (4 digits), study
number (3 digits), series number (3 digits), station number (3 digits), and
point number (4 digits).  Only the first three are needed to identify a study.
For example, \texttt{D-1914-001} is a study identifier.  The dashes are used
for readability but not stored in the database.
\texttt{D-1911-001-005-001-0013} is a point identifier.  Using these kind of
identifiers, it is possible to uniquely find data in the database.  Note that
the identifiers are systematic: the point identifier specifies the study,
series, and station it belongs to.

The flow classes are described in an appendix.


\section{Design goals}

\begin{itemize}

\item

Use SQLite as the database backend.  The goal here is to ensure that it
is possible to use SQL commands to select and find data.

The identifiers are intentionally simple and systematic so that it is possible
to extract information about profiles from the points (and in other
combinations).

\item

Use the Python package called \texttt{uncertainties} to handle uncertainties.
The uncertainties will be standard uncertainties (standard deviations of the
distribution of ``reasonable'' values from the measurements).

\item

Store the data as generic records.  This means that each data point in a flow
is merely a record in a particular table depending on its type.  This design is
flexible; it can store different data for the same variable easily, including
data for different averaging systems and measurement techniques.

Group data into 4 main categories:

    \begin{enumerate}

    \item Study data (data that applies to an entire study).

    \item Series data (data that applies to a series of profiles, like drag
    coefficients, etc.).

    \item Station data (data that applies to a single profile, like the
    momentum thickness)

    \item Point data (data that applies to a single point in a flow, like the
    local unweighted averaged velocity)

    \end{enumerate}

\item Eliminate redundancy in the tables.  Each data point MUST have one and
only one location.

Remember that at least some of the ``structure'' here really is ``data''.
Prefer data over structure, since data is mutable (or at least more easily
mutable).

Specifying the measurement techniques allows for all assumptions underlying
that data to be specified as well for each record in the database.

\item

Use assertions and other checks on the data.

\item

Include additional fields for commentary (likely editorial commentary).  Also
include additional fields for the people involved, the facilities involved, the
organizations involved, locations of any experiments, and any contract numbers.

\end{itemize}

